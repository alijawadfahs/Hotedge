\documentclass[10pt,twocolumn,letterpaper]{article}

\usepackage[english]{babel}
\usepackage[utf8x]{inputenc}
\usepackage[T1]{fontenc}


\usepackage[a4paper,top=3cm,bottom=2cm,left=3cm,right=3cm,marginparwidth=1.75cm]{geometry}

\usepackage{cite}
\usepackage{amsmath}
\usepackage{graphicx}
\usepackage[colorinlistoftodos]{todonotes}
\usepackage[colorlinks=true, allcolors=blue]{hyperref}
\usepackage{xcolor}
\newcommand{\note}[1]{\noindent{\color{red}\textbf{#1}}}
\definecolor{jazzberryjam}{rgb}{0.65, 0.04, 0.37}
\definecolor{ultramarine}{rgb}{0.07, 0.04, 0.56}
\newcommand{\notegp}[1]{\note{\color{jazzberryjam}[\textsc{Guillaume:} #1]}}
\newcommand{\noteaf}[1]{\note{\color{ultramarine}[\textsc{Ali:} #1]}}


%% Title
\title{
        %\vspace{-1in}  
        \usefont{OT1}{bch}{b}{n}
        \normalfont \normalsize \textsc{CSID, First Year Report} \\ [10pt]
        \huge  Decentralized Fog Computing Infrastructure Control \\
}

\usepackage{authblk}

\author{Ali J. FAHS \\
Supervised by Professor Guillaume PIERRE}

\date{May 18, 2018}


    \affil{\small{Univ Rennes, Inria, CNRS, IRISA}}

%%%%%%%%%%%%%%%%%%%%%%%%%% BEGIN DOCUMENT %%%%%%%%%%%%%%%%%%%%%%%%%%%%%%%%%%%%%%%%%%
\begin{document}
\maketitle

\selectlanguage{english}

\section{Introduction}
The main objective of this research work is creating the optimized infrastructure control for fog computing architecture. This broad objective is divided into steps that each will achieve a small improvement. 

During the course of 7 months, we have achieved a noticeable progress in shedding the light on the problem, and the lack of some key features that define the fog in the nowadays potential platforms. 

After the literature and defining the problem, we started the execution of a roadmap were we expect at the end of this map reaching a platform that can support the fog needs. 

\section{Fog platform \\ \& Kubernetes }

A fog computing platform is in charge of managing compute, storage and networking resources that are physically distributed across a geographical area such as a building, a neighborhood and a city ~\cite{bonomi2014, fogecosystem}.

The most prominent cloud platforms are Docker Swarm, Apache Mesos and Kubernetes~\cite{burns2016borg}, which all provide advanced features for container scheduling, cluster management, and support for large-scale clusters. They are currently the most mature systems to support the deployment of a fog/cloud platform. However, they lack a number of functionalities that are essential for large-scale fog computing scenarios, as we discuss in
the next section in the specific case of Kubernetes.


We have chosen to base our efforts on Kubernetes.  In particular, it offers many advanced functionalities such as the concepts of pods and services, auto-scaling, and a clean separation between the applications and the underlying platform. It has also attracted a very large community of users and developers, which positions it as a highly mature platform. However, Kubernetes was designed with cloud computing scenarios in mind. As a result the way Kubernetes manages the fog is far from good, we can demonstrate this by the following: 
\begin{itemize}
\item The way Kubernetes services distribute the work-load over containers replica set ignores the locations of the nodes, and  respectively the end-to-end latency.
\item The Deployment controller locates the users' application containers regardless of the nodes location. Such approach will not affect a cloud cluster but will have devastating outcomes for the fog. 
\item The infrastructure control in Kubernetes is completely centralized, such centralization imposes heavy network traffic for a minimal fog network. 
\end{itemize}

For the current time, we are digging deep in Kubernetes (2.2M lines of Go code + 1.2M lines in other languages). We are exposing the key elements of the platform, and about to implement a new service using iptables, that will support location awareness. 

\section{Work Progress}
\begin{enumerate}
\item Creating a Rpi's cluster that runs Kuberentes, that will be used as our test-bed. 
\item Configuring a local cluster for Kubernetes development.
\item Understanding the vital parts of the Kubernetes architecture in general, an the networking part in particular.
\item Submitting a towards paper to HotEdge 2018 workshop\cite{ali}, this paper contains a detailed roadmap for the upcoming steps. 
  
\end{enumerate}


\section*{Conclusions}

Fog and cloud computing platforms share many identical concerns, which
makes it attractive to use mature cloud computing platforms to build
new fog computing platforms. However, this requires a number of
adjustments in particular to support location-awareness. We have
chosen Kubernetes as the basis for our work, and proposed a roadmap of
the necessary adjustments before it can realistically be considered as
the world's best fog computing platform.

\bibliography{mybib}{}
\bibliographystyle{plain}


\end{document}